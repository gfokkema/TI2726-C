\documentclass[12pt]{article}

\begin{document}

\begin{titlepage}

\newcommand{\HRule}{\rule{\linewidth}{0.5mm}} % Defines a new command for the horizontal lines, change thickness here

\center % Center everything on the page
 
%----------------------------------------------------------------------------------------
%	HEADING SECTIONS
%----------------------------------------------------------------------------------------

\textsc{\LARGE Delft University of Technology}\\[1.5cm] % Name of your university/college
\textsc{\Large Operating Systems Laboratory}\\[0.5cm] % Major heading such as course name
\textsc{\large TI2726-C}\\[0.5cm] % Minor heading such as course title

%----------------------------------------------------------------------------------------
%	TITLE SECTION
%----------------------------------------------------------------------------------------

\HRule \\[0.4cm]
{ \huge \bfseries Bonus Question 4}\\[0.4cm] % Title of your document
\HRule \\[1.5cm]
 
%----------------------------------------------------------------------------------------
%	AUTHOR SECTION
%----------------------------------------------------------------------------------------

\begin{minipage}{0.4\textwidth}
\begin{flushleft} \large
\emph{Author:}\\
Gerlof \textsc{Fokkema} % Your name
\end{flushleft}
\end{minipage}
~
\begin{minipage}{0.4\textwidth}
\begin{flushright} \large
\emph{Lecturer:} \\
Prof.dr.ir. H.J. \textsc{Sips} % Supervisor's Name
\end{flushright}
\end{minipage}\\[1cm]

\begin{minipage}{0.4\textwidth}
\begin{flushleft} \large
\emph{Lab Assistants:}\\
Arjen \textsc{Rouvoet}\\
Ben \textsc{Allen}\\
Alex \textsc{van Rijs} % Your name
\end{flushleft}
\end{minipage}
~
\begin{minipage}{0.4\textwidth}
\begin{flushright} \large

\end{flushright}
\end{minipage}\\[3cm]

% If you don't want a supervisor, uncomment the two lines below and remove the section above
%\Large \emph{Author:}\\
%John \textsc{Smith}\\[3cm] % Your name

%----------------------------------------------------------------------------------------
%	DATE SECTION
%----------------------------------------------------------------------------------------

{\large \today}\\[3cm] % Date, change the \today to a set date if you want to be precise

%----------------------------------------------------------------------------------------
%	LOGO SECTION
%----------------------------------------------------------------------------------------

%\includegraphics{Logo}\\[1cm] % Include a department/university logo - this will require the graphicx package
 
%----------------------------------------------------------------------------------------

\vfill % Fill the rest of the page with whitespace

\end{titlepage}

\section*{Question 1}
The following scheduling algorithms can result in starvation:
\begin{itemize}
\item \textbf{Shortest job first} \\
      Suppose a very long job $j$ is scheduled at a time $t$.
      Now suppose that after $t$ a lot of short jobs are continuously scheduled. \\
      Since the shortest job will always be prioritized, our initial job $j$ will never get a chance to run.
\item \textbf{Priority} \\
      Suppose a job $t$ is scheduled at a time $t$ with a low priority $p$. \\
      Now suppose that after $t$ a lot of jobs are continuously scheduled with priority $q$, where  $q > p$.
      Again our initial job $j$ will never get a chance to run.
\end{itemize}

\section*{Question 2}
\begin{itemize}
\item \textbf{a} \\
  That process would be allowed to run twice as much as processes with only 1 pointer in the queue.
\item \textbf{b} \\
  \textbf{Advantages}
  \begin{itemize}
    \item Threads can be assigned priorities with round robin scheduling
    \item While this strategy supports priorities, it does not cause starvation
  \end{itemize}
  \textbf{Disadvantages}
  \begin{itemize}
    \item In order to remove a process from the queue, multiple entries have to be removed
    \item This strategy can still slow down execution of lower priority processes significantly
  \end{itemize}
\item \textbf{c} \\
  Assign a weight to each process, that will determine for how long a process will be allowed to run.
  This way processes with higher weights, will be allotted more execution time.
\end{itemize}

\section*{Question 3}\label{results}
\begin{itemize}
\item \textbf{a} \\
  The time quantum for a thread with priority 15 is 160. \\
  The time quantum for a thread with priority 40 is 40. \\
  Priority and the length of the assigned time quantum have an inverse relationship.
  The higher the priority is, the smaller the time quantum will be.
  This allows processes with high priority to be responsive,
  while allowing fast throughput for processes with low priority (and larger time quantum).
\item \textbf{b} \\
  Since this thread will now be considered CPU-intensive, it's priority will be lowered to 40.
  Having it's priority lowered, this process will now be allotted larger quanta.
\item \textbf{c} \\
  Assuming that there's new I/O availabe, it's new priority will be 52.
  This allows for responsiveness for interactive processes.
\end{itemize}

\end{document}