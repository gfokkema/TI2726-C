\documentclass[12pt]{article}

\begin{document}

\begin{titlepage}

\newcommand{\HRule}{\rule{\linewidth}{0.5mm}} % Defines a new command for the horizontal lines, change thickness here

\center % Center everything on the page
 
%----------------------------------------------------------------------------------------
%	HEADING SECTIONS
%----------------------------------------------------------------------------------------

\textsc{\LARGE Delft University of Technology}\\[1.5cm] % Name of your university/college
\textsc{\Large Operating Systems Laboratory}\\[0.5cm] % Major heading such as course name
\textsc{\large TI2726-C}\\[0.5cm] % Minor heading such as course title

%----------------------------------------------------------------------------------------
%	TITLE SECTION
%----------------------------------------------------------------------------------------

\HRule \\[0.4cm]
{ \huge \bfseries Question 5.4}\\[0.4cm] % Title of your document
\HRule \\[1.5cm]
 
%----------------------------------------------------------------------------------------
%	AUTHOR SECTION
%----------------------------------------------------------------------------------------

\begin{minipage}{0.4\textwidth}
\begin{flushleft} \large
\emph{Author:}\\
Gerlof \textsc{Fokkema} % Your name
\end{flushleft}
\end{minipage}
~
\begin{minipage}{0.4\textwidth}
\begin{flushright} \large
\emph{Lecturer:} \\
Prof.dr.ir. H.J. \textsc{Sips} % Supervisor's Name
\end{flushright}
\end{minipage}\\[1cm]

\begin{minipage}{0.4\textwidth}
\begin{flushleft} \large
\emph{Lab Assistants:}\\
Arjen \textsc{Rouvoet}\\
Ben \textsc{Allen}\\
Alex \textsc{van Rijs} % Your name
\end{flushleft}
\end{minipage}
~
\begin{minipage}{0.4\textwidth}
\begin{flushright} \large

\end{flushright}
\end{minipage}\\[3cm]

% If you don't want a supervisor, uncomment the two lines below and remove the section above
%\Large \emph{Author:}\\
%John \textsc{Smith}\\[3cm] % Your name

%----------------------------------------------------------------------------------------
%	DATE SECTION
%----------------------------------------------------------------------------------------

{\large \today}\\[3cm] % Date, change the \today to a set date if you want to be precise

%----------------------------------------------------------------------------------------
%	LOGO SECTION
%----------------------------------------------------------------------------------------

%\includegraphics{Logo}\\[1cm] % Include a department/university logo - this will require the graphicx package
 
%----------------------------------------------------------------------------------------

\vfill % Fill the rest of the page with whitespace

\end{titlepage}

\section*{Other possible methods}
Explain other methods that could be used to solve assignment 5.3. Why and how should those methods be used? \\

Since a few years multicore systems have become increasingly common.
Below we will highlight some of the alternative solutions that could be used.
\subsection*{Transactional memory}
In order to solve the synchronization problem on multicore systems, one could implement some sort
of transactional memory operations.
These operations would then guarantuee the user that an operation will be safely executed and that it leaves the resulting memory in the correct state. \\
The construct \textit{atomic\{S\}} mentioned in the book is an example of such an implementation.

\subsection*{OpenMP}
Another approach is to use the OpenMP compiler directives to define regions that can be run parallel and
critical sections.
This is in fact almost the same mechanism as just using mutexes, but it removes the need for defining mutexes by hand and is therefore slightly easier to use.

\subsection*{Functional Programming Languages}
Yet another approach is to use functional programming languages.
Since functional programming languages do not allow mutable state or mutable variables,
there is no way a program can run into synchronization issues when altering a variable,
since the whole concept of \textit{altering a value} does not exist.

\end{document}